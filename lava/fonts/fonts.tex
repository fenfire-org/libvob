\documentclass{article}
\usepackage{concmath}
\usepackage{euler}
\usepackage{bbm}
\begin{document}

\section{Ad hoc Ideas}


\subsection{Darkening}

Using "twice the ink", i.e.~doubling the alpha from an alpha text 
texture or calculating $1-2(1-x)$ for intensity makes
letters look crisper and better.

Assumes trilinear filtering is used.

Status: In use; helps somewhat.

\subsection{Subpixel rendering by texture accesses}

Status: Works.


\subsection{Distance filtering}

For bi-level functions, let the texture value be the minimum
distance (in texture space) to a texel of different color.
Filter using nearest-neighbour texture access mode and comparing
the distance to the pixel size.

Status: Seems to work if parameters chosen carefully.

\subsection{Distance and direction filtering}

In addition to the distance, store its gradient direction
(or access the texture to find out). Use to antialias
oblique ($45^o$) edges more than nearly horizontal or vertical ones.

Status: unknown

\subsection{LOD bias}

(A commonly used texture, trading temporal aliasing for spatial crispness)

Status: unknown for text

\subsection{Background whitening}

Adjust the background texture's color according
to a blurred sample from the foreground texture. Makes the text
stand out clearer.

Status: tested, ambiguous results

\subsection{Background blurring}

Adjust the background texture's bias or derivatives according
to a blurred sample from the foreground texture. Makes the background
"move away" from interfering with the foreground.

Status: unknown

%------------------------------------------
\section{Rigorous treatment}

Notation: Original function: $f(x) : \mathbbmss{R}^2 \rightarrow [0,1]$.  
Pixellated function: $g(p) : \mathbbmss{N}^2 \rightarrow [0,1]$.

Goal: Given $f$, choose $g$ to minimize $d(g, f)$ where $d$ 
is a distance function between a pixellated and normal function.
Then, choose $g$ \emph{efficiently} and \emph{locally}.

Two types of $f$: continuous an bi-level functions, both with
some intrinsic frequency distribution and feature width constraints.
Correlation functions??

Tasks:
\begin{itemize}
\item Define rigorously some distributions of unfiltered $f$ functions.
\item Define rigorously a good set of $d$ functions. To include: 
    simple integrals, 
\item Calculate each $d$ for each $f$ for mipmaps and summed-area tables (how many accesses!!!) and
    for various distributions of transformations.
\item Examine distance-based filtering for bi-level functions.
\end{itemize}

\subsection{Sets of unfiltered functions}

Can be defined through fourier transforms (bad) or various
correlation statistics.

\subsection{Distance functions}

Simple squared distance of area integrals.

Various frequency measure / correlation statistics. ?? !!!

Perceptual issues

For example, a vertical line of one pixel width should always remain $(0,0,0,1,0,0,0)$
and never $(0,0,0,.55,.45,0,0)$ by shifting the line away from being
in the middle of the pixels.
However, when the line is sloping, the jaggies will have to be dealt with in that way.

\subsection{Calculations}

Lots of integrals of distance function distributions.
Analytic or \emph{Monte Carlo}. For Monte Carlo, could use 
GPU for integrations!!

\subsection{Research questions / goals}

First: focus on bi-level functions; start with lines and curves.

Can distance filtering be useful?

Can we measure filtering performance satisfactorily efficiently?

Can precalculation of correlation statistics be used advantageously in
filtering font-like shapes, for decreasing global error metrics? 

Is aliasing easier to understand and control through correlation functions
than fourier transforms?!?!

\end{document}


